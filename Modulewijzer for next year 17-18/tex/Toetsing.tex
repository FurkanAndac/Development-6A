\section{Assessment}
	The course is tested with two exams: a practical assessment and a theoretical examination. The final grade is determined by the practical assessment. However, to be admitted to the practical assessment, you \textbf{must} have a sufficient (i.e. $\geq$ 5.5) grade in the theoretical examination.

	\subsection{Theoretical examination}
	The theoretical examination consists of a \textbf{written exam} which covers the topics seen in class. The questions will be both theoretical and about code analysis, such as understanding what a code snippet does, determining its complexity, or finding mistakes in it.
	The exam lasts two lesson hours (100 minutes). No help is allowed during the exam.\\
	A template of the exam can be found in Attachment 1.

	\subsection{Practical assessment}
	The practical examination is a \textbf{practical assessment} based on a \textbf{programming assignment} (to verify the authorship of code). In the programming assignment, some of the data structures and algorithms seen in class will have to be implemented and applied to a specific case study.\\ 
	A detailed description of the programming assignment can be found in Attachment 2.
	For the practical assessment you must be able to re-implement some parts of the programming assignment.
%	Important notes:
%	\begin{itemize}
%	\item The assignment must be done individually.
%	\item You must upload your project on Github and (only at the end) on N@tschool. The teachers must be added to the Github repository. 
%	\item Each exercise of the practical assignment is associated to a specific deadline. The intermediate deadlines will be checked through the commits in Github.
%	\item There will be oral checks to verify the authorship of code.
%	\item The framework for the assignment comes only for .NET languages: allowed languages are C\# and F\#.
%	\end{itemize}
	
	\subsection{Retake (herkansing)}
	If one part of the assessment is not sufficient (theoretical and/or practical examination), then you can repeat that part in the following block:
	\begin{itemize}
	\item In week 10 of the following block you can repeat the written exam.
	\item In week 10 of the following block you can repeat the practical assessment (if the written exam has been already passed).
	\end{itemize}
	
