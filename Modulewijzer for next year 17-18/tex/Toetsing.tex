\section{Assessment}
	The course is tested with two exams: a practical assessment and a theoretical examination. The final grade is determined by the practical assessment. However, to be admitted to the practical assessment, you \textbf{must} have a sufficient (i.e. $\geq$ 5.5) grade in the theoretical examination.

	\subsection{Theoretical examination}
	The theoretical examination consists of a \textbf{written exam} which covers the topics seen in class. The questions will be both theoretical and about code analysis, such as understanding what a code snippet does, determining its complexity, or finding mistakes in it.
	The exam lasts two lesson hours (100 minutes). No help is allowed during the exam.\\

	\subsection{Practical assessment}
	The practical examination is a \textbf{practical assessment} during which the student is asked to implement (completely or in part) algorithms seen during the course or strictly connected to course topics. 
	
	As preparation for the assessment, the students are strongly suggested to complete a \textbf{formative programming assignment}. In this programming assignment, some of the data structures and algorithms seen in class will have to be implemented and applied to a specific case study.\\ 
	A detailed description of the programming assignment can be found in Attachment 1.
	
	\subsection{Retake (herkansing)}
	If one part of the assessment is not sufficient (theoretical and/or practical examination), then you can repeat that part in the following educational period:
	\begin{itemize}
	\item In week 10 of the following OP you can repeat the written exam.
	\item In week 10 of the following OP you can repeat the practical assessment (if the written exam has been already passed).
	\end{itemize}
	
