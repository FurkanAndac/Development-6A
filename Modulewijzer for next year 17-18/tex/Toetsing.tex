\section{Assessment}
	The course is tested with two exams: a practical assessment and a theoretical examination. The final grade is determined by the practical assessment. However, to be admitted to the practical assessment, you \textbf{must} have a sufficient (i.e. $\geq$ 5.5) grade in the theoretical examination. If your grade in the theoretical examination is not sufficient, then we will register a 0 (zero) in Osiris as grade of the practical assessment.
	
	The correspondence between learning goals and assessment parts is shown in the following table. For more details, see the examination matrix in Attachment 1. \\
	
	\begin{table}[h]
		\centering
	\begin{tabular}{ |l|l|l| }
		\hline
		\textbf{Learning goal} & \textbf{Theory} & \textbf{Practice} \\
		\hline
		\texttt{PERF} & V & \\
		\hline 
		\texttt{DS}\textsuperscript{A} & V & \\
		\hline 
		\texttt{SORT}\textsuperscript{I} & & V \\
		\hline 
		\texttt{SORT}\textsuperscript{A} & V & \\
		\hline 
		\texttt{REC}\textsuperscript{I} & & V \\
		\hline 
		\texttt{REC}\textsuperscript{A} & V &  \\
		\hline 
		\texttt{GRAPH}\textsuperscript{I} & & V \\
		\hline 
		\texttt{GRAPH}\textsuperscript{A} & V & \\
		\hline 
	\end{tabular}
	\end{table}

	\subsection{Theoretical examination}
	The theoretical examination consists of a \textbf{written exam} which covers the topics seen in class. The questions will be both theoretical and about code analysis, such as understanding what a code snippet does, determining its complexity, or finding mistakes in it.
	The exam lasts two lesson hours (100 minutes). No help is allowed during the exam.\\
	
	The exam consists of: 
	\begin{itemize}
		\item 6 questions about complexity of algorithms (``What is the complexity of the following code/algorithm?")
		\item 4 questions about inference of the behaviour of a given algorithm (for example, ``What does this algorithms do?\textquotedblright{} or ``Does this algorithm find the minimum element?\textquotedblright, etc.)
	\end{itemize}

	\subsection{Practical assessment}
	The practical examination is a \textbf{practical assessment} during which the student is asked to implement (completely or in part) algorithms seen during the course or strictly connected to course topics. The practical assessment consists of:
	\begin{itemize}
		\item 5 assignments about filling in code of given partial algorithms
	\end{itemize}
	
	The programming language used for the practical assessment will be one of the languages covered by a previous Development course. More details will be given during the lessons.\\
	
	As preparation for the assessment, the students are strongly suggested to complete a \textbf{formative programming assignment} (see Attachment 2). In this programming assignment, some of the data structures and algorithms seen in class will have to be implemented and applied to a specific case study.\\ 

	
	\subsection{Retake (herkansing)}
	If one part of the assessment is not sufficient (theoretical and/or practical examination), then you can repeat that part in the following educational period:
	\begin{itemize}
	\item In week 10 of the following OP you can repeat the written exam.
	\item In week 10 of the following OP you can repeat the practical assessment (if the written exam has been already passed).
	\end{itemize}
	
