\section{Assessment}
	The course is tested with two exams: a practical assessment and a theoretical examination. The final grade is determined by the practical assessment. However, to pass the course, you \textbf{must} have a sufficient (i.e. $\geq$ 5.5) grade in the theoretical examination (as well as in the practical assessment).
	
	The correspondence between learning goals and assessment parts is shown in the following table. \\
	
	\begin{table}[h]
		\centering
	\begin{tabular}{ |l|l|l| }
		\hline
		\textbf{Learning goal} & \textbf{Theory} & \textbf{Practice} \\
		\hline
		\texttt{PERF} & V & \\
		\hline 
		\texttt{DS}\textsuperscript{A} & V & \\
		\hline 
		\texttt{SORT}\textsuperscript{I} & & V \\
		\hline 
		\texttt{SORT}\textsuperscript{A} & V & \\
		\hline 
		\texttt{REC}\textsuperscript{I} & & V \\
		\hline 
		\texttt{REC}\textsuperscript{A} & V &  \\
		\hline 
		\texttt{GRAPH}\textsuperscript{I} & & V \\
		\hline 
		\texttt{GRAPH}\textsuperscript{A} & V & \\
		\hline 
	\end{tabular}
	\end{table}

	\subsection{Theoretical examination}
	The theoretical examination consists of a \textbf{written exam} which covers the topics seen in class. The questions will be both theoretical and about code analysis. No help is allowed during the exam.\\
	
	The exam consists of multiple-choice questions about:
	\begin{itemize}
		\item complexity of algorithms (i.e., ``What is the complexity of the following code/algorithm?")
		\item inference of the behaviour of a given algorithm (i.e., ``What does this algorithms do?'', ``Does this algorithm find the minimum element?'', ``What is the next step of this algorithm given the following state?'', etc.)
		\item definitions given during the course
	\end{itemize}

	\noindent
	Each correct answer is worth 1 point. The grade of the written exam is computed as the percentage of correct answers in a scale from 0 to 10. For example, if the exam is made of 20 questions and you answer correctly 14 questions, then your grade is 14 * 10 / 20 = 7.

	\subsection{Practical assessment}
	The practical examination is a \textbf{practical assessment} during which the student is asked to implement (completely or in part) algorithms seen during the course or strictly connected to course topics. The practical assessment consists of a few assignments asking to fill in code of given partially-implemented algorithms. No help is allowed during the assessment.
	
	The programming language used for the practical assessment will be C\#. You need to install Visual Studio Community or Code on your laptop before the practical assessment. You must be able to compile C\# with the latest version of .NET Framework or .NET Core.
	
	As preparation for the assessment, the students are strongly suggested to implement on their own every algorithm and data structure presented in the course (see also the ``Homework'' sections on the lesson slides).
	%As preparation for the assessment, the students are strongly suggested to complete a \textbf{formative programming assignment} (see Attachment 1). In this programming assignment, some of the data structures and algorithms seen in class will have to be implemented and applied to a specific case study.

	
	\subsection{Retake (herkansing)}
	If only one of the exam parts is passed (theory or practice), then only the other will need to be retaken.