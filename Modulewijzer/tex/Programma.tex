\section{Course program}
	
	In the following table you can see the program of the course, divided in lesson units. Each lesson unit is also associated with the corresponding book paragraphs. The last lesson unit of the course is reserved for a summary in preparation for the exam. \\
	Note: Lesson units are intented as collections of topics and do not necessarily correspond to study weeks (for example, a lesson unit could span two study weeks). \\
	
	\begin{tabular}{ | p{1.2cm} | p{10cm} | p{2.7cm} | }
		\hline
	  	\textbf{Lesson unit} & \textbf{Topics} & \textbf{Book chapters and paragraphs} \\
	  	\hline
  		1 & Introduction to algorithms \newline Arrays \newline Complexity of algorithms (empirical analysis, O notation) & 3 \\
  		\hline
  		2 & Sorting algorithms \newline - Insertion sort \newline - Merge sort & 2.1, 2.3 \\
  		\hline
  		3 & List \newline Queue \newline Stack \newline Hash table &  10.1, 10.2, 11.1 until 11.4 \\
  		\hline
  		4 & Trees \newline - BST \newline - k-d trees \newline - 2-3 trees & 12.1 until 12.3, 18 \\
  		\hline
  		5 & Graphs \newline - undirected \newline - directed \newline - Dijkstra's shortest path & 22.1 until 22.3, 24.3 \\
  		\hline
  		6 & Dynamic programming \newline Floyd-Warshall & 15.3, 25.2\\
  		\hline
  		7 & Course review and preparation for the exam & \\
  		\hline
	\end{tabular}
\begin{comment}
  		4 & Binary search trees & 3.2 \\
  		\hline
  		5 & Balanced search trees: 2-3 search trees & 3.3 \\
  		\hline
  		6 & Graphs (undirected; directed; Dijkstra shortest path) & 4.1, 4.2, 4.4 \\
  		\hline
  		7 & Dynamic programming; Floyd-Warshall & Not covered by the book, see slides or Cormen\\
  		\hline
  		8 & Course recap & \\
  		\hline
\end{comment}

\noindent
\\
After each lesson unit, a multiple-choice questionnaire on the topics seen in class will be published. The questions are similar to those of the written exam.